% Created 2021-06-19 Sat 02:17
% Intended LaTeX compiler: pdflatex
\documentclass[11pt]{article}
\usepackage[utf8]{inputenc}
\usepackage[T1]{fontenc}
\usepackage{graphicx}
\usepackage{grffile}
\usepackage{longtable}
\usepackage{wrapfig}
\usepackage{rotating}
\usepackage[normalem]{ulem}
\usepackage{amsmath}
\usepackage{textcomp}
\usepackage{amssymb}
\usepackage{capt-of}
\usepackage{hyperref}
\usepackage[utf8x]{inputenc}
\usepackage[T2A]{fontenc}
\hypersetup{colorlinks, citecolor=black, filecolor=black, linkcolor=black, urlcolor=black}
\author{q}
\date{\today}
\title{Матан. Подготвка к экзамену.}
\hypersetup{
 pdfauthor={q},
 pdftitle={Матан. Подготвка к экзамену.},
 pdfkeywords={},
 pdfsubject={},
 pdfcreator={Emacs 27.2 (Org mode 9.4.4)}, 
 pdflang={English}}
\begin{document}

\maketitle
\tableofcontents

\section{Даты}
\label{sec:org0372a88}
\subsection{Консультация}
\label{sec:org0e2f668}
\emph{2021-06-24 Thu}
\subsection{Экзамен}
\label{sec:org8b9ddb3}
\emph{2021-06-25 Fri}

\section{Темы}
\label{sec:org1bfc3e4}
\subsection{Первообразная и неопределенный интеграл (определения). Свойства интеграла. Таблица основных неопределенных интегралов. Формула замены переменной в неопределенном интеграле (с доказательством). Формула интегрирования по частям.}
\label{sec:orgb11110c}

\subsubsection{Опр. 1.}
\label{sec:orgef1519a}
Функция \emph{F} называется первообразной функции \emph{f} на промежутке \(\Delta\) , если \emph{F} дифференцируема на \(\Delta\) и в каждой точке \emph{x} \(\in\) \(\Delta\)
\begin{eqnarray}
F'(x)=f(x)
\end{eqnarray}
Очевидно, что первообразная \emph{F(x)} непрерывна на \(\Delta\).

\subsubsection{Опр. 2.}
\label{sec:orgd132d37}
Пусть функция \emph{f(x)} задана на промежутке \(\Delta\). Совокупность всех ее первообразных на этом промежутке называется \emph{\textbf{неопределенным интегралом от функции \emph{f}}} и обозначается

\begin{eqnarray}
\int f(x)dx
\end{eqnarray}

Если \emph{F(x)} — какая-либо первообразная функции \emph{f(x)} на \(\Delta\), то пишут

\begin{eqnarray}
\int f(x)dx=F(x)+C
\end{eqnarray}

\emph{C} — произвольная постоянная.

\subsubsection{Основные свойства интеграла}
\label{sec:org08ee254}

\begin{enumerate}
\item Если функция \emph{F(x)} дифференцируема на \(\Delta\), то
\label{sec:orgd4cbf75}

\begin{eqnarray}
\int dF(x)=F(x)+C \text{ или }\int F'(x)dx=F(x)+C
\end{eqnarray}

\item Пусть функция \emph{f(x)} имеет первообразную на \(\Delta\). Тогда для любого \emph{x} \(\in\) \(\Delta\) имеет место равенство:
\label{sec:orge776878}

\begin{eqnarray}
d\int f(x)=f(x)dx
\end{eqnarray}

\item Если функции \emph{f\textsubscript{1}}, \emph{f\textsubscript{2}} имеют первообразные на \(\Delta\), то функция \emph{f\textsubscript{1}} + \emph{f\textsubscript{2}} имеет первообразную на \(\Delta\), причем:
\label{sec:org13f4fcb}

\begin{eqnarray}
\int(f_1(x) + f_2(x))dx=\int f_1(x)dx + \int f_2(x)dx
\end{eqnarray}

\item Если функция \emph{f(x)} имеет первообразную на \(\Delta\), \emph{k} \(\in\) \emph{\R}, то функция \emph{kf(x)} также имеет на \(\Delta\) первообразную, и при \emph{k} \(\ne\) 0:
\label{sec:org44c3bb2}

\begin{gather*}
\int kf(x)dx=\{kF(x)+C\}\text{, }k\int f(x)dx=k\{F(x)+C\}
\end{gather*}

Т.к. \emph{C} – произвольная постоянная и \emph{k} \(\ne\) 0, то множества \({kF(x) + C}\) и \(k{F(x) + C}\) совпадают.
\end{enumerate}

\subsubsection{След. 1 (Линейность интеграла)}
\label{sec:org1e3207a}
Если f\textsubscript{1} и f\textsubscript{2} имеют первообразные на \(\Delta\), \(\lambda\)\textsubscript{1}, \(\lambda_2 \in \R\), \(\lambda_1^2+\lambda_2^2>0\), 
то функция \(\lambda_1 f_1+\lambda_2 f_2\) имеет первообразную на \(\Delta\), причем

\begin{eqnarray}
\int(\lambda_1 f_1(x)+\lambda_2 f_2(x))dx=\lambda_1\int f_1(x)dx+\lambda_2\int f_2(x))dx
\end{eqnarray}

Доказательство вытекает из свойств 3 и 4.
\subsubsection{Формула замены переменной}
\label{sec:orgb60a726}
Пусть функции \(f(x)\) и \(\varphi(t)\) заданы соответственно на промежутках \(\Delta_x\) и \(\Delta_t\), 
причем \(\varphi (\Delta_t) = \Delta_x\), т.е. имеет смысл сложная функция \(f(\varphi(t))\), \(t \in \Delta_t\). 
Пусть, кроме того, функция \(\varphi(t)\) дифференцируема и строго монотонна на \(\Delta_t\). Тогда у функции \(\vatphi(t)\)
существует обратная однозначная функция \(\varphi^{-1}(x)\), определенная на промежутке \(\Delta_x\).

\textbf{Теорема 1.} Существование на промежутке \(\Delta_x\) интеграла

  \begin{eqnarray}
\int f(x)dx
  \end{eqnarray}

и существование на промежутке \(\Delta_t\) интеграла

  \begin{eqnarray}
\int f(\varphi(t))\varphi'(t)dt
  \end{eqnarray}

равносильны, и имеет место формула

  \begin{eqnarray}
\int f(x)dx=\int f(\varphi(t))\varphi'(t)dt\bigg|_{t=\varphi^{-1}(x)}
  \end{eqnarray}

Формула (10) называется формулой замены переменной в неопределенном интеграле:
переменная x заменяется переменной t по формуле \(x = \varphi(t)\).


\textbf{Доказательство.} Докажем, что существование первообразной у функции f(x) на
\(\Delta\)\textsubscript{x} равносильно существованию первообразной у функции \(f(\varphi(t))\varphi'(t)\)  на \(\Delta\)\textsubscript{t}
. Пусть у функции f(x) на \(\Delta\)\textsubscript{x} существует первообразная F(x), т.е.

\begin{eqnarray}
\frac{dF(x)}{dx} = f(x)\text{, } x\in\Delta_t
\end{eqnarray}

Имеет смысл сложная функция F (\(\varphi\)(t)), она является первообразной функции \(f(varphi(t))\varphi'(t)\) на \(\Delta\)\textsubscript{t}. 
Действительно,
d
dtF (ϕ(t)) = d F(x)
dx
¯
¯
¯
¯
x=ϕ(t)
d ϕ(t)
dt = f (ϕ(t)) ϕ
0
(t). (14)
Обратно. Пусть функция f (ϕ(t)) ϕ
0
(t) имеет первообразную Φ(t), тогда
d Φ(t)
dt = f (ϕ(t)) ϕ
0
(t). (15)
Покажем, что Φ (ϕ
−1
(x)) является на ∆x первообразной функции f(x). В самом
деле,
d
dxΦ
¡
ϕ
−1
(x)
¢
=
d Φ(t)
dt
¯
¯
¯
¯
t=ϕ−1(x)
d ϕ−1
(x)
dx = (f (ϕ(t)) ϕ
0
(t))|
t=ϕ−1(x) ×
×
dϕ−1
(x)
dx = f(x).
Итак, интегралы (10) и (11) одновременно существуют или нет. При этом
Z
f(x)dx = F(x) + C, (16)
Z
f (ϕ(t)) ϕ
0
(t)dt = F (ϕ(t)) + C,
а так как F (ϕ(t))|
t=ϕ−1(x) = F(x), имеет равенство (12).
\subsection{Определенный интеграл Римана (определение). Ограниченность интегрируемых функций (с доказательством). Верхние и нижние суммы Дарбу (определения). Верхний и нижний интегралы Дарбу (определения). Критерий Дарбу. Интегрируемость непрерывных функций. Интегрируемость монотонных функций.}
\label{sec:org3cdacc4}

\subsection{Свойства определенного интеграла (сформулировать все, доказать непрерывность интеграла по верхнему пределу). Интегральная теорема о среднем.}
\label{sec:org2faf248}

\subsection{Теорема о дифференцировании интеграла по верхнему пределу (с доказательством).  Теорема о существовании первообразной (с доказательством). Формула Ньютона-Лейбница (с доказательством). Формула замены переменной в определенном интеграле. Формула интегрирования по частям.}
\label{sec:org7a26360}

\subsection{Определение несобственных интегралов.  Формула Ньютона-Лейбница и формула замены переменной для несобственных интегралов.}
\label{sec:orgea14674}

\subsection{Несобственные интегралы от неотрицательных функций (лемма и признак сравнения). Критерий Коши сходимости интеграла (с доказательством). Абсолютно сходящиеся интегралы (определение и теорема о сходимости абсолютно сходящегося интеграла).}
\label{sec:org40e6d97}

\subsection{Определение числового ряда. Необходимый признак сходимости ряда (с доказательством). Критерий Коши сходимости ряда (с доказательством). Ряды с неотрицательными членами (признак сравнения, интегральный признак Коши, радикальный признак Коши, признак Даламбера).}
\label{sec:orgfdc7d9a}

\subsection{Знакопеременные ряды (признак Лейбница). Абсолютно сходящиеся ряды (определение). Критерий Коши абсолютной сходимости ряда. Условно сходящиеся ряды (определение). Теорема Римана.}
\label{sec:org438ae31}

\subsection{Функциональные последовательности  и ряды (определения, в том числе, ограниченная последовательность, сходящаяся последовательность, сходящийся ряд, абсолютно сходящийся ряд). Равномерная сходимость функциональной последовательности и функционального ряда (определение и пример). Критерии Коши равномерной сходимости функциональной последовательности (ряда). Признак Вейерштрасса.}
\label{sec:orgc2de2a3}

\subsection{Свойства равномерно сходящихся рядов (непрерывность суммы (с доказательством), интегрирование, дифференцирование).}
\label{sec:org2b9ce89}

\subsection{Степенные ряды (определение). Первая теорема Абеля (с доказательством). Радиус и круг (интервал) сходимости степенного ряда (определения). Понятие аналитической функции (определение). Теорема о представлении аналитической функции рядом Тейлора.}
\label{sec:orgd892d60}

\subsection{Определение n-мерного арифметического евклидова пространства. Определение n-мерного открытого шара. Предел последовательности в n-мерном пространстве, ограниченное множество  в n-мерном пространстве, окрестность бесконечно удалённой точки (определения).}
\label{sec:org2646d9e}

\subsection{Внутренняя точка множества, открытое множество, точка прикосновения множества, предельная точка множества, замыкание множества, замкнутое множество, компактное множество, линейно связное множество, выпуклое множество, область (определения).}
\label{sec:orgd6cb203}
\end{document}
